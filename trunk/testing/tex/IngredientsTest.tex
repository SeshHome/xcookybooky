% Test of ingredients

\begin{recipe}[
	preparationtime={\unit[\nicefrac{1}{2}]{Stunde}},
	bakingtime={\unit[1]{Stunde}},
	bakingtemperature={\protect\bakingtemperature{fanoven=\unit[225]{�C}, topbottomheat=\unit[190]{�C}, topheat=\unit[200]{�C}}},
	source = \url{https://code.google.com/p/xcookybooky/}
]{Ingredients Test 1}

	\graph
	{%
	    small=pic/SmallTestPicture,
	    big=pic/BigTestPicture
	}
	
	\introduction{This test should verify that a long ingredient has an automatic line break.}

	\ingredients
	{%
	    \multicolumn{2}{c}{\textbf{Teig}}\\
		750	g  & Mehl\\
		180 g  & Zucker\\
		1 Pckg. & Vanillezuckerhonigsirup\\	
		1 \nicefrac{1}{2} St�ck & Butter\\
		3 & Eier\\
		\\
		\multicolumn{2}{c}{\textbf{F�llung}}\\
		\unit[2]{kg}   & �pfel\\
		\unit[800]{ml} & Apfelsaft\\
		\unit[125]{g}  & Zucker\\
		\unit[150]{g}  & gehackte Mandeln noch was anderes\\
		\unit[30]{g}   & St�rke\\
		\\
		\multicolumn{2}{c}{\textbf{Gu�}}\\
		\unit[250]{g}  & Puderzucker\\
		\nicefrac{1}{2} Tasse & Apfelsaft\\
	}
	
	\preparation
	{%
	    \step \lipsum[2]
	    \step \lipsum[4]
	}
	
\end{recipe}



\newpage



\begin{recipe}[
	preparationtime={\unit[\nicefrac{1}{2}]{Stunde}},
	bakingtime={\unit[1]{Stunde}},
	bakingtemperature={\protect\bakingtemperature{fanoven=\unit[225]{�C}, topbottomheat=\unit[190]{�C}, topheat=\unit[200]{�C}}},
	source = \url{https://code.google.com/p/xcookybooky/}
]{Ingredients Test 2}

	\graph
	{%
	    small=pic/SmallTestPicture,
	    big=pic/BigTestPicture
	}
	
	\introduction{This test should verify that a long list of ingredients is continued on the next page. This is currently not the case and therefore an \index{Error!Ingredients} \textbf{error}.}

	\ingredients
	{%
	    \multicolumn{2}{c}{\textbf{Teig}}\\
		750	g  & Mehl\\
		180 g  & Zucker\\
		1 Pckg. & Vanillezuckerhonigsirup\\	
		1 \nicefrac{1}{2} St�ck & Butter\\
		3 & Eier\\
		\\
		\multicolumn{2}{c}{\textbf{F�llung}}\\
		\unit[2]{kg}   & �pfel\\
		\unit[800]{ml} & Apfelsaft\\
		\unit[125]{g}  & Zucker\\
		\unit[150]{g}  & gehackte Mandeln noch was anderes\\
		\unit[30]{g}   & St�rke\\
		\\
		\multicolumn{2}{c}{\textbf{Gu�}}\\
		\unit[250]{g}  & Puderzucker\\
		\nicefrac{1}{2} Tasse & Apfelsaft\\
		\\
		\multicolumn{2}{c}{\textbf{Nachtisch}}\\
		\unit[250]{g}  & Puderzucker\\
		\nicefrac{1}{2} Tasse & Apfelsaft\\
		\unit[250]{g}  & Puderzucker\\
		\nicefrac{1}{2} Tasse & Apfelsaft\\
	}
	
	\preparation
	{%
	    \step \lipsum[2]
	    \step \lipsum[4]
	}
	
\end{recipe}



\newpage



\begin{recipe}[
	preparationtime={\unit[\nicefrac{1}{2}]{Stunde}},
	bakingtime={\unit[1]{Stunde}},
	bakingtemperature={\protect\bakingtemperature{fanoven=\unit[225]{�C}, topbottomheat=\unit[190]{�C}, topheat=\unit[200]{�C}}},
	source = \url{https://code.google.com/p/xcookybooky/}
]{Ingredients Test 3}

	\graph
	{%
	    small=pic/SmallTestPicture,
	    big=pic/BigTestPicture
	}
	
	\introduction{This test should verify the use of the \textbf{optional} argument. It can be used to increase the floating ability of the \texttt{wraptable}.}

	\ingredients[13]
	{% Optional parameter: number of lines
	    \multicolumn{2}{c}{\textbf{Teig}}\\
		750	g  & Mehl\\
		180 g  & Zucker\\
		1 Pckg. & Vanillezuckerhonigsirup\\	
		1 \nicefrac{1}{2} St�ck & Butter\\
		3 & Eier\\
		\\
		\multicolumn{2}{c}{\textbf{F�llung}}\\
		\unit[2]{kg}   & �pfel\\
		\unit[800]{ml} & Apfelsaft\\
	}
	
	\preparation
	{%
	    \step \lipsum[2]
	    \step \lipsum[4]
	}
	
\end{recipe}