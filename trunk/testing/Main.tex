\documentclass[%
a4paper,
%twoside,
%12pt
11pt
]{article}


\usepackage[T1]{fontenc}
\usepackage[latin1]{inputenc} 
\usepackage{pbsi} % font package
\usepackage{lmodern} % font package
\usepackage{bookman} % font package
\usepackage
[% Language test
	ngerman,
	french,
	spanish,
	english
]{babel}
\usepackage{nicefrac}
\usepackage{booktabs} % commands for tabulars
\usepackage
[%
%	dvipsnames,
%	svgnames,
	x11names,
]{xcolor}
\usepackage{blindtext}
\usepackage{lipsum}
\usepackage{makeidx}

\usepackage[
	%handwritten,
	%myconfig,
	nowarnings
]{xcookybooky}

%% DEFINE COLORS
%\definecolor{DarkGray}{rgb}{0.23, 0.23, 0.23}

\newcommand*{\rect}[2][0.8]
{% Creates an rectangle filles with the specified color.
\begin{tikzpicture}[scale=#1]
	\draw[fill=#2] (0,0) -- ++(up:1em) -- ++(right:2em) -- ++(down:1em) -- cycle;
	\draw[draw=black] (0,0) -- ++(up:1em) -- ++(right:2em) -- ++(down:1em) -- cycle;
\end{tikzpicture}
}


\DeclareRobustCommand{\textcelcius}{\ensuremath{^{\circ}\mathrm{C}}}

\setcounter{secnumdepth}{0} % No numeration of chapters
\renewcommand{\recipesection}[2][]
{% Recipes are subsections (instead of subsubsections)
    \subsection[#1]{#2}
}

\renewcommand{\subsectionmark}[1]
{% Not displayed in order to show the section name
}

\renewcommand{\sectionmark}[1]
{% Uppercased section names
	\markright{\MakeUppercase{#1}}
}

\title{Testing the package xcookybooky}
\author{Sven Harder}

%%%%%%%%%%
% hyperref
\usepackage{hyperref}	% must be loaded at last!
\hypersetup	{%
	pdfauthor			= {Sven Harder},
	pdftitle 			= {Testing the xcookybooky package},
	pdfsubject			= {testing},
	pdfkeywords			= {recipes},
	pdfstartview		= {FitV}, 	
	pdfview				= {FitH},
	pdfpagemode			= {UseOutlines}, % Options; UseNone, UseOutlines
	bookmarksopen		= {true},
	pdfpagetransition	= {Glitter},
	colorlinks			= {true},
	linkcolor			= {black}, % internal links in the document, e.g. to a figure
	urlcolor			= {black}, % Links to the internet
	citecolor			= {black}, % Links to citings
	filecolor			= {black}, % Links to local files
}
% hyperref
%%%%%%%%%%
\hbadness=10000	% Ignore underfull boxes

% Eliminate the page break
\makeatletter
\renewenvironment{theindex}
{%
	\section*{\indexname}%
    \@mkboth{\MakeUppercase\indexname}{\MakeUppercase\indexname}%
    \parskip\z@ \@plus .3\p@\relax
    \columnseprule \z@
    \columnsep 35\p@
    \let\item\@idxitem}
{}
\makeatother

\makeindex


\begin{document}

% Title page and table of contents
\thispagestyle{empty}

\maketitle

\tableofcontents

\newpage

\setBackgroundPicture%[x=2.2cm, y=-0.8cm, width=\linewidth, height=\paperheight]
[x, y, width=\paperwidth-3cm, height, orientation=pagecenter]
{pic/background}


\section{Results}
The following index is used to find the detected errors in this test document. Using MiKTeX~2.9 no errors, no warnings and no overfull boxes should be reported.
\printindex



\section{Basic Tests}
This section contains tests for the elementary layout commands. The example recipe is included as a reference.

%% Example

\begin{otherlanguage}{ngerman}

\setHeadlines
{% translation
	inghead = Zutaten,
    prephead = Zubereitung,
    hinthead = Tipp,
    continuationhead = Fortsetzung,
    continuationfoot = Fortsetzung auf n\"achster Seite,
    portionvalue = Personen,
}

\begin{recipe}
[ % Optionale Eingaben
	preparationtime = {\unit[1]{h}},
	portion = \portion{5},
	source = R. Gaus
]
{Mousse au Chocolat}
	
	\graph
	{% Bilder
		small=pic/glass,	% kleines Bild
		big=pic/ingredients % gro�es (l�ngeres) Bild
	}
	
	\ingredients
	)\\
		3		 & Eier\\
		\unit[200]{ml} & Sahne\\
		\unit[40]{g} & Zucker\\
		\unit[50]{g} & Butter
	}
	
	\preparation
	{ % Zubereitung
		\step Eier trennen, Eiwei� und Sahne separat steif schlagen. Butter und Schokolade vorsichtig im Wasserbad schmelzen.
		\step Eigelb in einer gro�en Sch�ssel mit \unit[2]{EL} hei�em Wasser cremig schlagen, den Zucker einr�hren bis die Masse hell und cremig ist.
		\step Die geschmolzene Schokolade unterheben, anschlie�end sofort Eischnee und Sahne unterheben (nicht mit dem Elektro-Mixer!)
		\step Mindestens 2 Stunden im K�hlschrank kalt stellen. Aber nicht zu kalt servieren.
	}
	
	\hint
	{%
		Der Schokoladenanteil kann auch gesenkt werden.
	}

\end{recipe}

\end{otherlanguage}

% Test of graph (pictures)

\begin{recipe}[
	preparationtime={\unit[\nicefrac{1}{2}]{hour}},
	bakingtime={\unit[1]{hour}},
	source = http://www.ctan.org/pkg/cookingsymbols
]{Graph Test 1}

	\graph
	{%
	    small=pic/SmallTestPicture,
	}
	
	\introduction{This test checks if the pictures are set correct. In this test case only the \textbf{small} picture is set.}

	\ingredients
	{%
	    \multicolumn{2}{c}{\textbf{Something}}\\
		180  & Whatever
	}
	
	\preparation
	{%
	    \step \lipsum[2]
	    \step \lipsum[4]
	}

\end{recipe}



\newpage



\begin{recipe}[
	preparationtime={\unit[\nicefrac{1}{2}]{hour}},
	bakingtime={\unit[1]{hour}},
	source = http://www.ctan.org/pkg/cookingsymbols
]{Graph Test 2}

	\graph
	{%
	    big=pic/BigTestPicture,
	}
	
	\introduction{This test checks if the pictures are set correct. In this test case only the \textbf{big} picture is set.}

	\ingredients
	{%
	    \multicolumn{2}{c}{\textbf{Something}}\\
		180  & Whatever
	}
	
	\preparation
	{%
	    \step \lipsum[2]
	    \step \lipsum[4]
	}

\end{recipe}



\newpage



\begin{recipe}[
	preparationtime={\unit[\nicefrac{1}{2}]{hour}},
	bakingtime={\unit[1]{hour}},
	source = http://www.ctan.org/pkg/cookingsymbols
]{Graph Test 3}
	
	\introduction{This test checks if the pictures are set correct. In this test case \textbf{no} pictures are set.}

	\ingredients
	{%
	    \multicolumn{2}{c}{\textbf{Something}}\\
		180  & Whatever
	}
	
	\preparation
	{%
	    \step \lipsum[2]
	    \step \lipsum[4]
	}

\end{recipe}



\newpage



\begin{recipe}[
	preparationtime={\unit[\nicefrac{1}{2}]{hour}},
	bakingtime={\unit[1]{hour}},
	source = http://www.ctan.org/pkg/cookingsymbols
]{Graph Test 4}
	
	\graph
    {% pictures
        small=pic/SmallTestPicture,
        big=pic/BigTestPicture,
		smallpicturewidth = 0.6\textwidth,
		bigpicturewidth=0.3\textwidth,
    }
	
	\introduction{This test checks if the pictures are set correct. In this test case the width of the pictures is changed only for this recipe. Therefore the next recipe should have the default values again.}

	\ingredients
	{%
	    \multicolumn{2}{c}{\textbf{Something}}\\
		180  & Whatever
	}
	
	\preparation
	{%
	    \step \lipsum[2]
	    \step \lipsum[4]
	}

\end{recipe}

% Test of recipe title

\begin{recipe}[
	preparationtime={\unit[\nicefrac{1}{2}]{Stunde}},
	portion = \portion{10},
	source = \url{https://code.google.com/p/xcookybooky/}
]{Recipe Title Test which has a long name for a simple test}

	\graph
	{%
	    small=pic/SmallTestPicture,
	    big=pic/BigTestPicture,
	}
	
	\introduction{This test is intended to verify that the recipe title can be longer than one line.}

	\ingredients
	{%
		750	g  & Something \\
		180 g  & a bit of that
	}
	
	\preparation
	{%
	    \step \lipsum[2]
	    \step \lipsum[4]
	}
	
\end{recipe}

% Test of recipe overview

\begin{recipe}[
	preparationtime={\unit[\nicefrac{1}{2}]{hour}},
	bakingtime={\unit[1]{hour}},
	bakingtemperature={\protect\bakingtemperature{fanoven=\unit[225]{�C}, topbottomheat=\unit[190]{�C}, topheat=\unit[200]{�C}}},
	portion = \portion{10},
	calory= {\unit[100]{kJ}},
	source = \url{https://code.google.com/p/xcookybooky/}
]{Recipe Overview Test 1}

	\graph
	{%
	    small=pic/SmallTestPicture,
	    big=pic/BigTestPicture,
	}
	
	\introduction{This test is used to check different configurations of the recipe overview. In this test all possible entries are set.}

	\ingredients
	{%
		\unit[1]{bottle} & Rum
	}
	
	\preparation
	{%
	    \step \lipsum[2]
	    \step \lipsum[4]
	}
	
\end{recipe}



\newpage



\begin{recipe}[
	bakingtime={\unit[1]{hour}},
	calory= {\unit[100]{kJ}},
	source = \url{https://code.google.com/p/xcookybooky/}
]{Recipe Overview Test 2}

	\graph
	{%
	    small=pic/SmallTestPicture,
	    big=pic/BigTestPicture,
	}
	
	\introduction{This test is used to check different configurations of the recipe overview. In this test the \textbf{baking time} is set without the \textbf{baking temperature} among other things.}

	\ingredients
	{%
		\unit[1]{bottle} & Rum
	}
	
	\preparation
	{%
	    \step \lipsum[2]
	    \step \lipsum[4]
	}
	
\end{recipe}



\newpage



\begin{recipe}[
	bakingtemperature={\protect\bakingtemperature{fanoven=\unit[225]{�C}, topbottomheat=\unit[190]{�C}, topheat=\unit[200]{�C}}},
	portion = \portion{10},
	source = \url{https://code.google.com/p/xcookybooky/}
]{Recipe Overview Test 3}

	\graph
	{%
	    small=pic/SmallTestPicture,
	    big=pic/BigTestPicture,
	}
	
	\introduction{This test is used to check different configurations of the recipe overview. In this test the \textbf{baking temperature} is set without the \textbf{baking time} among other things.}

	\ingredients
	{%
		\unit[1]{bottle} & Rum
	}
	
	\preparation
	{%
	    \step \lipsum[2]
	    \step \lipsum[4]
	}
	
\end{recipe}



\newpage



\begin{recipe}[
	preparationtime={\unit[\nicefrac{1}{2}]{hour}},
	portion = \portion{10}
]{Recipe Overview Test 4}

	\graph
	{%
	    small=pic/SmallTestPicture,
	    big=pic/BigTestPicture,
	}
	
	\introduction{This test is used to check different configurations of the recipe overview. This test shows the \index{Error!Recipe Overview} \textbf{error}, that the gap to the introduction frame is larger if the source is \textbf{not} set.}

	\ingredients
	{%
		\unit[1]{bottle} & Rum
	}
	
	\preparation
	{%
	    \step \lipsum[2]
	    \step \lipsum[4]
	}
	
\end{recipe}


% Test of introduction

\begin{recipe}[
	preparationtime={\unit[\nicefrac{1}{2}]{hour}},
	bakingtime={\unit[1]{hour}},
	source = http://www.ctan.org/pkg/xcookybooky
]{Introduction Test 1}

	\graph
	{%
	    small=pic/SmallTestPicture,
	    big=pic/BigTestPicture
	}
	
	\introduction{This test checks the behaviour if the introduction is very long.
	
	\blindtext
	
	\blindtext
	
	\blindtext
	
	\blindtext
	}

	\ingredients
	{%
	    \multicolumn{2}{c}{\textbf{Something}}\\
		180  & Whatever
	}
	
	\preparation
	{%
	    \step \lipsum[2]
	    \step \lipsum[4]
	}
	
	\hint
	{%
		\blindtext
	}

\end{recipe}



\newpage



\begin{recipe}[
	preparationtime={\unit[\nicefrac{1}{2}]{hour}},
	bakingtime={\unit[1]{hour}},
	source = http://www.ctan.org/pkg/cookingsymbols
]{Introduction Test 2}

	\graph
	{%
	    small=pic/SmallTestPicture,
	    big=pic/BigTestPicture
	}

	\ingredients
	{%
	    \multicolumn{2}{c}{\textbf{Something}}\\
		180  & Whatever
	}
	
	\preparation
	{%
	    \step This test is used to check the introduction. In this test case \textbf{no} introduction is set.
	}
	
\end{recipe}

% Test of preparation

\begin{recipe}[
	preparationtime={\unit[5]{min}},
	portion=\portion{4},
	source={http://www.ctan.org/pkg/xcookybooky}
]{Preparation Test 1}

	\graph
	{%
	    small=pic/SmallTestPicture,
	    big=pic/BigTestPicture,
	}
	
	\introduction{This test is used to verify the setting of preparation steps. In this test case the second preparation step has a page break inside.}

	\ingredients
	{%
	    \nicefrac{1}{2} l & Water\\
	    4 tsp & Rose hip tea\\
	    2 tbs & Honey\\
	    1    & Lemon\\
	    \nicefrac{1}{2} l & Fruit juice\\
	    \unit[100]{g}  & Strawberry\\
	    \unit[50]{g}  & Raspberry\\
	    1      & Orange\\
	    \unit[100]{g}  & Grapes\\
	}
	
	\preparation
	{%
	    \step \blindtext
	    \step \lipsum[1]
	    \step \blindtext
	}
	
	\hint
	{%
	    \blindtext
	}

\end{recipe}



\newpage



\begin{recipe}[
	preparationtime={\unit[5]{min}},
	portion=\portion{4},
	source={http://www.ctan.org/pkg/xcookybooky}
]{Preparation Test 2}

	\graph
	{%
	    small=pic/SmallTestPicture,
	    big=pic/BigTestPicture,
	}
	
	\introduction{This test is used to verify the setting of preparation steps. In this test case very short steps are used.}

	\ingredients
	{%
	    \nicefrac{1}{2} l & Water\\
	    4 tsp & Rose hip tea\\
	    2 tbs & Honey\\
	    1    & Lemon\\
	    \nicefrac{1}{2} l & Fruit juice\\
	    \unit[100]{g}  & Strawberry\\
	    \unit[50]{g}  & Raspberry\\
	    1      & Orange\\
	    \unit[100]{g}  & Grapes\\
	}
	
	\preparation
	{%
	    \step First step
	    \step Second step
	    \step The third preparation is little bit longer than two steps before.
	    \step Fourth step
	    \step Fifth step
	    \step Sixth step
	    \step Seventh step
	    \step Eighth step
	    \step Ninth step
	    \step Tenth step
	    \step Eleventh step
	    \step Twelfth step
	}
	
\end{recipe}



\newpage



\begin{recipe}[
	preparationtime={\unit[5]{min}},
	portion=\portion{4},
	source={http://www.ctan.org/pkg/xcookybooky}
]{Preparation Test 3}

	\graph
	{%
	    small=pic/SmallTestPicture,
	    big=pic/BigTestPicture,
	}
	
	\introduction{This test is used to verify the setting of preparation steps. In this test case the first preparation should float around the ingredients. But this is not the case and therefore an \index{Error!Preparation} \textbf{error}.}

	\ingredients
	{%
	    \nicefrac{1}{2} l & Water\\
	    4 tsp & Rose hip tea\\
	    2 tbs & Honey\\
	    1    & Lemon\\
	    \nicefrac{1}{2} l & Fruit juice\\
	    \unit[100]{g}  & Strawberry\\
	    \unit[50]{g}  & Raspberry\\
	    1      & Orange\\
	    \unit[100]{g}  & Grapes\\
	}
	
	\preparation
	{%
	    \step \blindtext\blindtext\blindtext
	    \step \blindtext
	}
	
	\hint
	{%
	    \blindtext
	}

\end{recipe}


\newpage


\begin{recipe}[
	preparationtime={\unit[5]{min}},
	portion=\portion{4},
	source={http://www.ctan.org/pkg/xcookybooky}
]{Preparation Test 4}

	\graph
	{%
	    small=pic/SmallTestPicture,
	    big=pic/BigTestPicture,
	}
	
	\introduction{This test is used to verify the setting of preparation steps. In this test case no steps are used.}

	\ingredients
	{%
	    \nicefrac{1}{2} l & Water\\
	    4 tsp & Rose hip tea\\
	    2 tbs & Honey\\
	    1    & Lemon\\
	    \nicefrac{1}{2} l & Fruit juice\\
	    \unit[100]{g}  & Strawberry\\
	    \unit[50]{g}  & Raspberry\\
	    1      & Orange\\
	    \unit[100]{g}  & Grapes\\
	}
	
	\preparation
	{%
	    \blinditemize
	    
	    \blindtext
	}
	
	\hint
	{%
	    \blindtext
	}

\end{recipe}

% Test of ingredients

\begin{recipe}[
	preparationtime={\unit[\nicefrac{1}{2}]{Stunde}},
	bakingtime={\unit[1]{Stunde}},
	bakingtemperature={\protect\bakingtemperature{fanoven=\unit[225]{�C}, topbottomheat=\unit[190]{�C}, topheat=\unit[200]{�C}}},
	source = \url{https://code.google.com/p/xcookybooky/}
]{Ingredients Test 1}

	\graph
	{%
	    small=pic/SmallTestPicture,
	    big=pic/BigTestPicture
	}
	
	\introduction{This test should verify that a long ingredient has an automatic line break.}

	\ingredients
	{%
	    \multicolumn{2}{c}{\textbf{Teig}}\\
		750	g  & Mehl\\
		180 g  & Zucker\\
		1 Pckg. & Vanillezuckerhonigsirup\\	
		1 \nicefrac{1}{2} St�ck & Butter\\
		3 & Eier\\
		\\
		\multicolumn{2}{c}{\textbf{F�llung}}\\
		\unit[2]{kg}   & �pfel\\
		\unit[800]{ml} & Apfelsaft\\
		\unit[125]{g}  & Zucker\\
		\unit[150]{g}  & gehackte Mandeln noch was anderes\\
		\unit[30]{g}   & St�rke\\
		\\
		\multicolumn{2}{c}{\textbf{Gu�}}\\
		\unit[250]{g}  & Puderzucker\\
		\nicefrac{1}{2} Tasse & Apfelsaft\\
	}
	
	\preparation
	{%
	    \step \lipsum[2]
	    \step \lipsum[4]
	}
	
\end{recipe}



\newpage



\begin{recipe}[
	preparationtime={\unit[\nicefrac{1}{2}]{Stunde}},
	bakingtime={\unit[1]{Stunde}},
	bakingtemperature={\protect\bakingtemperature{fanoven=\unit[225]{�C}, topbottomheat=\unit[190]{�C}, topheat=\unit[200]{�C}}},
	source = \url{https://code.google.com/p/xcookybooky/}
]{Ingredients Test 2}

	\graph
	{%
	    small=pic/SmallTestPicture,
	    big=pic/BigTestPicture
	}
	
	\introduction{This test should verify that a long list of ingredients is continued on the next page. This is currently not the case and therefore an \index{Error!Ingredients} \textbf{error}.}

	\ingredients
	{%
	    \multicolumn{2}{c}{\textbf{Teig}}\\
		750	g  & Mehl\\
		180 g  & Zucker\\
		1 Pckg. & Vanillezuckerhonigsirup\\	
		1 \nicefrac{1}{2} St�ck & Butter\\
		3 & Eier\\
		\\
		\multicolumn{2}{c}{\textbf{F�llung}}\\
		\unit[2]{kg}   & �pfel\\
		\unit[800]{ml} & Apfelsaft\\
		\unit[125]{g}  & Zucker\\
		\unit[150]{g}  & gehackte Mandeln noch was anderes\\
		\unit[30]{g}   & St�rke\\
		\\
		\multicolumn{2}{c}{\textbf{Gu�}}\\
		\unit[250]{g}  & Puderzucker\\
		\nicefrac{1}{2} Tasse & Apfelsaft\\
		\\
		\multicolumn{2}{c}{\textbf{Nachtisch}}\\
		\unit[250]{g}  & Puderzucker\\
		\nicefrac{1}{2} Tasse & Apfelsaft\\
		\unit[250]{g}  & Puderzucker\\
		\nicefrac{1}{2} Tasse & Apfelsaft\\
	}
	
	\preparation
	{%
	    \step \lipsum[2]
	    \step \lipsum[4]
	}
	
\end{recipe}



\newpage



\begin{recipe}[
	preparationtime={\unit[\nicefrac{1}{2}]{Stunde}},
	bakingtime={\unit[1]{Stunde}},
	bakingtemperature={\protect\bakingtemperature{fanoven=\unit[225]{�C}, topbottomheat=\unit[190]{�C}, topheat=\unit[200]{�C}}},
	source = \url{https://code.google.com/p/xcookybooky/}
]{Ingredients Test 3}

	\graph
	{%
	    small=pic/SmallTestPicture,
	    big=pic/BigTestPicture
	}
	
	\introduction{This test should verify the use of the \textbf{optional} argument. It can be used to increase the floating ability of the \texttt{wraptable}.}

	\ingredients[13]
	{% Optional parameter: number of lines
	    \multicolumn{2}{c}{\textbf{Teig}}\\
		750	g  & Mehl\\
		180 g  & Zucker\\
		1 Pckg. & Vanillezuckerhonigsirup\\	
		1 \nicefrac{1}{2} St�ck & Butter\\
		3 & Eier\\
		\\
		\multicolumn{2}{c}{\textbf{F�llung}}\\
		\unit[2]{kg}   & �pfel\\
		\unit[800]{ml} & Apfelsaft\\
	}
	
	\preparation
	{%
	    \step \lipsum[2]
	    \step \lipsum[4]
	}
	
\end{recipe}

% Test of hint

\begin{recipe}[
	bakingtime={\unit[1]{hour}},
	portion = \portion{10},
	source = \url{https://code.google.com/p/xcookybooky/}
]{Hint Test 1}

	\graph
	{%
	    small=pic/SmallTestPicture,
	    big=pic/BigTestPicture,
	}
	
	\introduction{This test should test the position of the hint. It must be at the bottom at the first page.}

	\ingredients
	{%
		1 & Unicorn \\
		\unit[2]{l} & Milk
	}
	
	\preparation
	{%
	    \step \lipsum[2]
	}
	
	\hint
	{%
	    \lipsum[4]
	}

\end{recipe}



\newpage



\begin{recipe}[
	bakingtime={\unit[1]{hour}},
	portion = \portion{10},
	source = \url{https://code.google.com/p/xcookybooky/}
]{Hint Test 2}

	\graph
	{%
	    small=pic/SmallTestPicture,
	    big=pic/BigTestPicture,
	}
	
	\introduction{This test should test the position of the hint. It must be at the \textbf{top} at the second page.}

	\ingredients
	{%
		1 & Unicorn \\
		\unit[2]{l} & Milk
	}
	
	\preparation
	{%
	    \step \lipsum[2]
	    \step \lipsum[4]
	}
	
	\hint
	{%
	    \lipsum[1]
	}

\end{recipe}



\newpage



\begin{recipe}[
	bakingtime={\unit[1]{Stunde}},
	portion = \portion{10},
	source = \url{https://code.google.com/p/xcookybooky/}
]{Hint Test 3}

	\graph
	{%
	    small=pic/SmallTestPicture,
	    big=pic/BigTestPicture,
	}
	
	\introduction{This test should test the position of the hint. It must be at the \textbf{bottom} of the second page.}

	\ingredients
	{%
		1 & Unicorn \\
		\unit[2]{l} & Milk
	}
	
	\preparation
	{%
	    \step \lipsum[2]
	    \step \lipsum[4]
	    \step \lipsum[7]
	}
	
	\hint
	{%
	    \lipsum[3]
	}

\end{recipe}



\newpage



\begin{recipe}[
	bakingtime={\unit[1]{Stunde}},
	portion = \portion{10},
	source = \url{https://code.google.com/p/xcookybooky/}
]{Hint Test 4}

	\graph
	{%
	    small=pic/SmallTestPicture,
	    big=pic/BigTestPicture,
	}
	
	\introduction{This test should test the behaviour of a long hint.}

	\ingredients
	{%
		1 & Unicorn \\
		\unit[2]{l} & Milk
	}
	
	\preparation
	{%
	    \step \lipsum[2]
	}
	
	\hint
	{%
	    \blindtext
	    
	    \blindtext
	    
	    \blindtext
	    
	    \blindtext
	    
	    \blindtext
	    
	    \blindtext
	}

\end{recipe}


\setBackgroundPicture%[x=2.2cm, y=-1.0cm, width=\linewidth, height=\paperheight]
[x, y, width=\paperwidth-3cm, height, orientation=pagecenter]
{pic/background}


\section{Advanced Tests}
This test section contains special tests for commands, which are not needed by many people.

%% Hook test

\begin{recipe}[
	preparationtime={\unit[5]{min}},
	portion=\portion{4},
	source={https://code.google.com/p/xcookybooky/}
]{Hook Test Recipe}

	\pregraph{\textbf{\textcolor{red}{PRE GRAPH}} Hook: It must be the first block in the recipe.}
	
	\pretitle{\textbf{\textcolor{red}{PRE TITLE}} Hook: It must be between the pictures and the title.}
	
	\prepreparation{\textbf{\textcolor{red}{PRE PREPARATION}} Hook: It must be inside the preparation block and \textbf{before} the preparation steps.}
	\postpreparation{\textbf{\textcolor{red}{POST PREPARATION}} Hook: It must be inside the preparation block and \textbf{after} the preparation steps.}
	
	\preingredients{\textbf{\textcolor{red}{PRE INGREDIENTS}} Hook: It must be inside the ingredients part and \textbf{before} the ingredients.}
	\postingredients{\textbf{\textcolor{red}{POST INGREDIENTS}} Hook: It must be inside the ingredients block and \textbf{after} the ingredients.}
	
	\graph
	{%
	    small=pic/SmallTestPicture,
	    big=pic/BigTestPicture,
	}%
	
	\introduction{This should test the introduction. \blindtext}

	\ingredients % Zutaten
	{%
	    \nicefrac{1}{2} l & Wasser\\
	    4 TL & Hagebuttentee\\
	    2 EL & Honig\\
	    1    & Zitrone\\
	    \nicefrac{1}{2} l & Fruchtsaft\\
	    100 g  & Erdbeeren\\
	    50  g  & Himbeeren\\
	    1      & Orange\\
	    100 g  & Weintrauben\\
	}
	
	\preparation % Zubereitung
	{%
	    \step \blindtext
	    \step \blindtext
	    \step \blindtext
	}
	
	\hint
	{%
	    \blindtext
	}

\end{recipe}

% 
\begin{recipe}[
	preparationtime={\unit[5]{min}},
	portion=\portion{4},
	source={http://www.ctan.org/pkg/xcookybooky}
]{Long Text Test}

	\graph
	{%
	    small=pic/SmallTestPicture,
	    big=pic/BigTestPicture,
	}%
	
	\introduction{This test is intended to test a very long recipe. It is also used to test the continuation texts.}

	\ingredients
	{%
	    \nicefrac{1}{2} l & Water\\
	    4 TL & Rose hip tea\\
	    2 EL & Honey\\
	    1    & Lemon\\
	    \nicefrac{1}{2} l & Fruit juice\\
	    100 g  & Strawberry\\
	    50  g  & Raspberry\\
	    1      & Orange\\
	    100 g  & Grapes\\
	}
	
	\preparation
	{%
	    \step \blindtext
	    \step \blindtext
	    \step \blindtext
	    \step \blindtext
	    \step \blindtext
	    \step \blindtext
	    \step \blindtext
	    \step \blindtext
	    \step \blindtext
	    \step \blindtext
	    \step \blindtext
	    \step \blindtext
	    \step \blindtext
	    \step \blindtext
	    \step \blindtext
	    \step \blindtext
	    \step \blindtext
	    \step \blindtext
	}
	
	\hint
	{%
	    \blindtext
	}

\end{recipe}



\section{Supporting Command Tests}
These commands are useful to insert optional information for the recipe.

% Test for baking temperature

\begin{recipe}[
	preparationtime={\unit[\nicefrac{1}{2}]{hour}},
	bakingtime={\unit[1]{hour}},
	bakingtemperature={\protect\bakingtemperature{fanoven=\unit[225]{\textcelcius}}},
	portion = \portion{10},
	source = Mister Nobody
]{Baking Temperature Test 1}
	\graph
	{%
		small=pic/SmallTestPicture,
		big=pic/BigTestPicture
	}
	
	\introduction{This test should verify the baking temperature command, especially the comma}
	
	\ingredients
	{%
		\multicolumn{2}{c}{\textbf{Headline}}\\
		\unit[2]{kg}   & Something \\
		\unit[15]{g}   & Anything \\
		\unit[\nicefrac{1}{2}]{l}   & Everything
	}
	
	\preparation
	{%
		\step \blindtext
	}
	
\end{recipe}



\newpage



\begin{recipe}[
	preparationtime={\unit[\nicefrac{1}{2}]{hour}},
	bakingtime={\unit[1]{hour}},
	bakingtemperature={\protect\bakingtemperature{topbottomheat=\unit[190]{�C}, gasstove=Stufe 2}},
	portion = \portion{10},
	source = Mister Nobody
]{Baking Temperature Test 2}
	\graph
	{%
		small=pic/SmallTestPicture,
		big=pic/BigTestPicture
	}
	
	\introduction{This test should verify the baking temperature command, especially the comma}
	
	\ingredients
	{%
		\multicolumn{2}{c}{\textbf{Headline}}\\
		\unit[2]{kg}   & Something \\
		\unit[15]{g}   & Anything \\
		\unit[\nicefrac{1}{2}]{l}   & Everything
	}
	
	\preparation
	{%
		\step \blindtext
	}
	
\end{recipe}



\newpage



\begin{recipe}[
	preparationtime={\unit[\nicefrac{1}{2}]{hour}},
	bakingtime={\unit[1]{hour}},
	bakingtemperature={\protect\bakingtemperature{fanoven=\unit[225]{\textcelcius}, topbottomheat=\unit[190]{�C}, topheat=\unit[200]{�C}, bottomheat=\unit[300]{�C}, gasstove=Stufe 3}},
	portion = \portion{10},
	source = Mister Nobody
]{Baking Temperature Test 3}
	\graph
	{%
		small=pic/SmallTestPicture,
		big=pic/BigTestPicture
	}
	
	\introduction{This test should verify the baking temperature command, especially the comma}
	
	\ingredients
	{%
		\multicolumn{2}{c}{\textbf{Headline}}\\
		\unit[2]{kg}   & Something \\
		\unit[15]{g}   & Anything \\
		\unit[\nicefrac{1}{2}]{l}   & Everything
	}
	
	\preparation
	{%
		\step \blindtext
	}
	
\end{recipe}

% Test for portions

\begin{recipe}[
	preparationtime={\unit[\nicefrac{1}{2}]{hour}},
	bakingtime={\unit[1]{hour}},
	bakingtemperature={\protect\bakingtemperature{fanoven=\unit[225]{\textcelcius}}},
	portion = {\portion[master pieces for being satisfied]{10}},
	source = Mister Nobody
]{Portion Test}
	\graph
	{%
		small=pic/SmallTestPicture,
		big=pic/BigTestPicture
	}
	
	\introduction{This test should verify the portion temperature command. In this test case the optional parameter is tested.}
	
	\ingredients
	{%
		\multicolumn{2}{c}{\textbf{Headline}}\\
		\unit[2]{kg}   & Something \\
		\unit[15]{g}   & Anything \\
		\unit[\nicefrac{1}{2}]{l}   & Everything
	}
	
	\preparation
	{%
		\step \blindtext
	}
	
\end{recipe}





\section{Layout Tests}
The following pages contain tests to verify the possibilities of changing the layout of the recipes.

% Test of recipe color management

\setRecipeColors
{% Colors from xcolor using the option x11names 
    recipename = DodgerBlue2,
    intro = Blue1,
    ing = Bisque4,
    inghead = DeepPink1,
    prep = Chartreuse1,
    prephead = Coral4,
    suggestion,
    hint = Goldenrod1,
    hinthead = DarkOrchid1,
    hintline = Cyan1,
    numeration = Salmon1
}


\begin{recipe}[
	preparationtime={\unit[5]{min}},
	portion=\portion{4},
	source={http://www.ctan.org/pkg/xcookybooky}
]{Color Test}

	\graph
	{%
	    small=pic/SmallTestPicture,
	    big=pic/BigTestPicture,
	}
	
	\introduction{This test is used to verify the setting of the recipe colors. In this test case the following colors are set. Note that the introduction has changed its color, that is an \index{Error!Introduction} \textbf{error}. 
		\begin{center}
			\begin{tabular}{lrl}
				\textbf{Component} & \textbf{Color} \\
				\toprule
				Recipe Name & light blue & \rect{DodgerBlue2} \\
				Introduction & blue & \rect{Blue1} \\
				Ingredients (text) & gray & \rect{Bisque4} \\
				Ingredients (headline) & pink & \rect{DeepPink1} \\
				Preparation (text) & light green & \rect{Chartreuse1} \\
				Preparation (headline) & brown & \rect{Coral4} \\
				Suggestion & & \\
				Hint (text) & dark yellow & \rect{Goldenrod1} \\
				Hint (headline) & purple & \rect{DarkOrchid1} \\
				Hint (line) & cyan & \rect{Cyan1} \\
				Step (numeration) & light red & \rect{Salmon1}
			\end{tabular}
		\end{center}
	}

	\ingredients
	{%
	    \nicefrac{1}{2} l & Water\\
	    4 tsp & Rose hip tea\\
	    2 tbs & Honey\\
	    1    & Lemon\\
	    \nicefrac{1}{2} l & Fruit juice\\
	    \unit[100]{g}  & Strawberry\\
	    \unit[50]{g}  & Raspberry\\
	    1      & Orange\\
	    \unit[100]{g}  & Grapes\\
	}
	
	\preparation
	{%
	    \step \lipsum[4]
	    \step \lipsum[32]
	}
	
	\hint
	{%
	    \blindtext
	}

\end{recipe}



\setRecipeColors
{% Reset to default color values
    recipename,
    intro,
    ing,
    inghead,
    prep,
    prephead,
    suggestion,
    hint,
    hinthead,
    hintline,
    numeration
}

% Test of recipe font management

\setRecipenameFont{pbsi}{T1}{xl}{n}

\begin{recipe}[
	preparationtime={\unit[5]{min}},
	portion=\portion{4},
	source={http://www.ctan.org/pkg/xcookybooky}
]{Font Test 1}

	\graph
	{%
	    small=pic/SmallTestPicture,
	    big=pic/BigTestPicture,
	}
	
	\introduction{This test is used to check the correct setting of the recipe name font. In this test case the font \textbf{pbsi} from the package \texttt{pbsi} is used.}

	\ingredients
	{%
	    \nicefrac{1}{2} l & Water\\
	    4 tsp & Rose hip tea\\
	    2 tbs & Honey\\
	    1    & Lemon\\
	    \nicefrac{1}{2} l & Fruit juice\\
	    \unit[100]{g}  & Strawberry\\
	    \unit[50]{g}  & Raspberry\\
	    1      & Orange\\
	    \unit[100]{g}  & Grapes\\
	}
	
	\preparation
	{%
	    \step \lipsum[4]
	}
	
	\hint
	{%
	    \lipsum[6]
	}

\end{recipe}



\newpage



\setRecipenameFont{fau}{T1}{m}{n} % emerald package

\begin{recipe}[
	preparationtime={\unit[5]{min}},
	portion=\portion{4},
	source={http://www.ctan.org/pkg/xcookybooky}
]{Font Test 2}

	\graph
	{%
	    small=pic/SmallTestPicture,
	    big=pic/BigTestPicture,
	}
	
	\introduction{This test is used to check the correct setting of the recipe name font. In this test case the font \textbf{fau} from the package \texttt{emerald} is used.}

	\ingredients
	{%
	    \nicefrac{1}{2} l & Water\\
	    4 tsp & Rose hip tea\\
	    2 tbs & Honey\\
	    1    & Lemon\\
	    \nicefrac{1}{2} l & Fruit juice\\
	    \unit[100]{g}  & Strawberry\\
	    \unit[50]{g}  & Raspberry\\
	    1      & Orange\\
	    \unit[100]{g}  & Grapes\\
	}
	
	\preparation
	{%
	    \step \lipsum[4]
	}
	
	\hint
	{%
	    \lipsum[6]
	}

\end{recipe}



\newpage



\setRecipenameFont{fwb}{T1}{m}{n} % emerald package

\begin{recipe}[
	preparationtime={\unit[5]{min}},
	portion=\portion{4},
	source={http://www.ctan.org/pkg/xcookybooky}
]{Font Test 3}

	\graph
	{%
	    small=pic/SmallTestPicture,
	    big=pic/BigTestPicture,
	}
	
	\introduction{This test is used to check the correct setting of the recipe name font. In this test case the font \textbf{fwb} from the package \texttt{emerald} is used.}

	\ingredients
	{%
	    \nicefrac{1}{2} l & Water\\
	    4 tsp & Rose hip tea\\
	    2 tbs & Honey\\
	    1    & Lemon\\
	    \nicefrac{1}{2} l & Fruit juice\\
	    \unit[100]{g}  & Strawberry\\
	    \unit[50]{g}  & Raspberry\\
	    1      & Orange\\
	    \unit[100]{g}  & Grapes\\
	}
	
	\preparation
	{%
	    \step \lipsum[4]
	}
	
	\hint
	{%
	    \lipsum[6]
	}

\end{recipe}



\newpage



\setRecipenameFont{fjd}{T1}{m}{n} % emerald package

\begin{recipe}[
	preparationtime={\unit[5]{min}},
	portion=\portion{4},
	source={http://www.ctan.org/pkg/xcookybooky}
]{Font Test 4}

	\graph
	{%
	    small=pic/SmallTestPicture,
	    big=pic/BigTestPicture,
	}
	
	\introduction{This test is used to check the correct setting of the recipe name font. In this test case the font \textbf{fjd} from the package \texttt{emerald} is used.}

	\ingredients
	{%
	    \nicefrac{1}{2} l & Water\\
	    4 tsp & Rose hip tea\\
	    2 tbs & Honey\\
	    1    & Lemon\\
	    \nicefrac{1}{2} l & Fruit juice\\
	    \unit[100]{g}  & Strawberry\\
	    \unit[50]{g}  & Raspberry\\
	    1      & Orange\\
	    \unit[100]{g}  & Grapes\\
	}
	
	\preparation
	{%
	    \step \lipsum[4]
	}
	
	\hint
	{%
	    \lipsum[6]
	}

\end{recipe}



% Reset to the default
\setRecipenameFont{\familydefault}{\encodingdefault}{b}{n}


% Test of recipe sizes management

\setRecipeSizes
{% 
    recipename = \small,
    intro = \Large,
    ing = \tiny,
    inghead = \Large,
    prep = \scriptsize,
    prephead = \huge,
    suggestion,
    hint = \large,
    hinthead = \small
}

\begin{recipe}[
	preparationtime={\unit[5]{min}},
	portion=\portion{4},
	source={http://www.ctan.org/pkg/xcookybooky}
]{Font Size Test}

	\graph
	{%
	    small=pic/SmallTestPicture,
	    big=pic/BigTestPicture,
	}
	
	\introduction{This test is intended to check the setting of the font sizes.}

	\ingredients
	{%
	    \nicefrac{1}{2} l & Water\\
	    4 tsp & Rose hip tea\\
	    2 tbs & Honey\\
	    1    & Lemon\\
	    \nicefrac{1}{2} l & Fruit juice\\
	    \unit[100]{g}  & Strawberry\\
	    \unit[50]{g}  & Raspberry\\
	    1      & Orange\\
	    \unit[100]{g}  & Grapes\\
	}
	
	\preparation
	{%
	    \step \blindtext
	    \step \lipsum[1]
	    \step \blindtext
	}
	
	\hint
	{%
	    \blindtext
	}

\end{recipe}



\setRecipeSizes
{% reset to default values
    recipename,
    intro,
    ing,
    inghead,
    prep,
    prephead,
    suggestion,
    hint,
    hinthead
}

% Test of recipe lengths

\setRecipeLengths
{%
    pictureheight = 4.5cm,
    bigpicturewidth = 0.49 \textwidth,
    smallpicturewidth = 0.49 \textwidth,
    introductionwidth = 0.8 \textwidth,
    preparationwidth = 0.5 \textwidth,
    ingredientswidth = 0.5 \textwidth
}

\begin{recipe}[
	preparationtime={\unit[5]{min}},
	portion=\portion{4},
	source={http://www.ctan.org/pkg/xcookybooky}
]{Length Test}

	\graph
	 of the text width. And last but not least the ingredient and preparation part should take the half of the text width. The introduction width is not adjusted and therefore an \index{Error!Introduction} \textbf{error}.}

	\ingredients
	{%
	    \nicefrac{1}{2} l & Water\\
	    4 tsp & Rose hip tea\\
	    2 tbs & Honey\\
	    1    & Lemon\\
	    \nicefrac{1}{2} l & Fruit juice\\
	    \unit[100]{g}  & Strawberry\\
	    \unit[50]{g}  & Raspberry\\
	    1      & Orange\\
	    \unit[100]{g}  & Grapes\\
	}
	
	\preparation
	{%
	    \step \blindtext
	    \step \blindtext
	}
	
	\hint
	{%
	    \blindtext
	}

\end{recipe}


\setRecipeLengths
{% Reset to default values
    pictureheight,
    bigpicturewidth,
    smallpicturewidth,
    introductionwidth,
    preparationwidth,
    ingredientswidth
}

% Test of recipe headlines management


\selectlanguage{ngerman}

\IfLanguagePatterns{ngerman}
{
    \setHeadlines
    {% translation
        inghead = Zutaten,
        prephead = Zubereitung,
        hinthead = Tipp,
        continuationhead = Fortsetzung,
        continuationfoot = Fortsetzung auf n\"achster Seite,
        portionvalue = Personen,
        calory = Brennwert
    }
}{}

\begin{recipe}[
	preparationtime={\unit[5]{min}},
	portion = \portion{5},
	calory = {\unit[153]{kJ}},
	source={http://www.ctan.org/pkg/xcookybooky}
]{Headline Test: German}

	\graph
	{%
	    small=pic/SmallTestPicture,
	    big=pic/BigTestPicture,
	}
	
	\introduction{This test is used to verify the setting of the headlines of the recipe. This test case uses the German translation.}

	\ingredients
	{%
	    \nicefrac{1}{2} l & Water\\
	    4 tsp & Rose hip tea\\
	    2 tbs & Honey\\
	    1    & Lemon\\
	    \nicefrac{1}{2} l & Fruit juice\\
	    \unit[100]{g}  & Strawberry\\
	    \unit[50]{g}  & Raspberry\\
	    1      & Orange\\
	    \unit[100]{g}  & Grapes\\
	}
	
	\preparation
	{%
	    \step \blindtext
	}
	
	\hint
	{%
	    \blindtext
	}

\end{recipe}



\newpage



\selectlanguage{french}

\IfLanguagePatterns{french}
{% French
    \setHeadlines
    {% translation
        inghead = Ingr�dients,
        prephead = Pr�paration,
        hinthead = Tuyau,
        continuationhead = Suite,
        continuationfoot = Suite page suivante,
        portionvalue = Portions,
        calory = Valeur calorifique
    }
}{}

\begin{recipe}[
	preparationtime={\unit[5]{min}},
	portion = \portion{5},
	calory = {\unit[153]{kJ}},
	source={http://www.ctan.org/pkg/xcookybooky
}]{Headline Test: French}

	\graph
	{%
	    small=pic/SmallTestPicture,
	    big=pic/BigTestPicture,
	}
	
	\introduction{This test is used to verify the setting of the headlines of the recipe. This test case uses the French translation.}

	\ingredients
	{%
	    \nicefrac{1}{2} l & Water\\
	    4 tsp & Rose hip tea\\
	    2 tbs & Honey\\
	    1    & Lemon\\
	    \nicefrac{1}{2} l & Fruit juice\\
	    \unit[100]{g}  & Strawberry\\
	    \unit[50]{g}  & Raspberry\\
	    1      & Orange\\
	    \unit[100]{g}  & Grapes\\
	}
	
	\preparation
	{%
	    \step \blindtext
	}
	
	\hint
	{%
	    \blindtext
	}

\end{recipe}



\newpage

\selectlanguage{spanish}

\IfLanguagePatterns{spanish}
{% Spanish
    \setHeadlines
    {% translation
        inghead = Ingredientes,
        prephead = Preparaci�n,
        hinthead = Soplo,
        continuationhead = Continuaci�n,
        continuationfoot = Contin�a en la p�gina siguiente,
        portionvalue = Porci�n,
        calory = Poder calor�fico
    }
}{}

\begin{recipe}[
	preparationtime={\unit[5]{min}},
	portion = \portion{5},
	calory = {\unit[153]{kJ}},
	source={http://www.ctan.org/pkg/xcookybooky
}]{Headline Test: Spanish}

	\graph
	{%
	    small=pic/SmallTestPicture,
	    big=pic/BigTestPicture,
	}
	
	\introduction{This test is used to verify the setting of the headlines of the recipe. This test case uses the Spanish translation. Unfortunately the package \texttt{blindtext} does not support Spanish.}

	\ingredients
	{%
	    \nicefrac{1}{2} l & Water\\
	    4 tsp & Rose hip tea\\
	    2 tbs & Honey\\
	    1    & Lemon\\
	    \nicefrac{1}{2} l & Fruit juice\\
	    \unit[100]{g}  & Strawberry\\
	    \unit[50]{g}  & Raspberry\\
	    1      & Orange\\
	    \unit[100]{g}  & Grapes\\
	}
	
	\preparation
	{%
	    \step \lipsum[5] % blindtext does not support spanish
	}
	
	\hint
	{%
	    \lipsum[5] % blindtext does not support spanish
	}

\end{recipe}



\newpage

\selectlanguage{portuguese}

\IfLanguagePatterns{portuguese}
{% Portuguese
	\setHeadlines
	{% translation
		inghead = Ingredientes,
		prephead = Prepara\c{c}\~{a}o,
		hinthead = Dica,
		continuationhead = Continua\c{c}\~{a}o,
		continuationfoot = Continua na pr\'{o}xima p\'{a}gina,
		portionvalue = Por\c{c}\~{o}es,
		calory = Valor Cal\'{o}rico
	}
}{}

\begin{recipe}[
	preparationtime={\unit[5]{min}},
	portion = \portion{5},
	calory = {\unit[153]{kJ}},
	source={http://www.ctan.org/pkg/xcookybooky
}]{Headline Test: Portuguese}

	\graph
	{%
	    small=pic/SmallTestPicture,
	    big=pic/BigTestPicture,
	}
	
	\introduction{This test is used to verify the setting of the headlines of the recipe. This test case uses the Portuguese translation. Unfortunately the package \texttt{blindtext} does not support Portuguese.}

	\ingredients
	{%
	    \nicefrac{1}{2} l & Water\\
	    4 tsp & Rose hip tea\\
	    2 tbs & Honey\\
	    1    & Lemon\\
	    \nicefrac{1}{2} l & Fruit juice\\
	    \unit[100]{g}  & Strawberry\\
	    \unit[50]{g}  & Raspberry\\
	    1      & Orange\\
	    \unit[100]{g}  & Grapes\\
	}
	
	\preparation
	{%
	    \step \lipsum[5] % blindtext does not support Portuguese
	}
	
	\hint
	{%
	    \lipsum[5] % blindtext does not support Portuguese
	}

\end{recipe}



\newpage

\selectlanguage{brazil}

\IfLanguagePatterns{brazil}
{% Portuguese PT-BR
	\setHeadlines
	{% translation
		inghead = Ingredientes,
		prephead = Prepara\c{c}\~{a}o,
		hinthead = Dica,
		continuationhead = Continua\c{c}\~{a}o,
		continuationfoot = Continua na pr\'{o}xima p\'{a}gina,
		portionvalue = Por\c{c}\~{o}es,
		calory = Valor Cal\'{o}rico
	}
}{}

\begin{recipe}[
	preparationtime={\unit[5]{min}},
	portion = \portion{5},
	calory = {\unit[153]{kJ}},
	source={http://www.ctan.org/pkg/xcookybooky
}]{Headline Test: Brazilian}

	\graph
	{%
	    small=pic/SmallTestPicture,
	    big=pic/BigTestPicture,
	}
	
	\introduction{This test is used to verify the setting of the headlines of the recipe. This test case uses the Brazilian translation. Unfortunately the package \texttt{blindtext} does not support Brazilian.}

	\ingredients
	{%
	    \nicefrac{1}{2} l & Water\\
	    4 tsp & Rose hip tea\\
	    2 tbs & Honey\\
	    1    & Lemon\\
	    \nicefrac{1}{2} l & Fruit juice\\
	    \unit[100]{g}  & Strawberry\\
	    \unit[50]{g}  & Raspberry\\
	    1      & Orange\\
	    \unit[100]{g}  & Grapes\\
	}
	
	\preparation
	{%
	    \step \lipsum[5] % blindtext does not support Brazilian
	}
	
	\hint
	{%
	    \lipsum[5] % blindtext does not support Brazilian
	}

\end{recipe}



\setHeadlines
{% reset to default
    inghead,
    prephead,
    hinthead,
    continuationhead,
    continuationfoot,
    portionvalue,
    calory
}


%
\end{document} 